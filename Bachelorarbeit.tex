%!TEX TS-program = pdflatex
%!TEX TS-options = -shell-escape
% Author: Carolin Wortmann
\documentclass[%
	paper=a4,
	fontsize=11pt,
	ngerman,
	twoside,
	toc=listof, 
	]{scrbook}

% :::::::::::::::::::::::::::::::::::::::::::::::::::::::::::
% ::::::::::::::::: Encoding/Language :::::::::::::::::::::::
% :::::::::::::::::::::::::::::::::::::::::::::::::::::::::::

	\usepackage{scrtime}	
	\usepackage{datetime}									% To set date	
	\usepackage{etex}
	\usepackage{shellesc}									% Compile using -shell-escape!
	\usepackage[utf8]{inputenc}								% UTF8-encoding
	\usepackage{babel}										% German typographical rules 
	\usepackage[german=quotes]{csquotes}					% German quotes 
	
	
% :::::::::::::::::::::::::::::::::::::::::::::::::::::::::::
% :::::::::::::::::::::: Colours ::::::::::::::::::::::::::::
% :::::::::::::::::::::::::::::::::::::::::::::::::::::::::::
	
	\usepackage[usenames,x11names,final]{xcolor}
	\definecolor{fbblau}{HTML}{3078AB}
	\definecolor{mediumgrey}{gray}{.65}
	\definecolor{darkgrey}{gray}{.35}	
	\definecolor{stdblue}{HTML}{00387A}
	\definecolor{stdred}{HTML}{900000}

	
% :::::::::::::::::::::::::::::::::::::::::::::::::::::::::::
% ::::::::::::::::::: Hyperrefs/URLS ::::::::::::::::::::::::
% :::::::::::::::::::::::::::::::::::::::::::::::::::::::::::	
	
	\usepackage{url}
	\usepackage[%
		hidelinks,
		pdfpagelabels,
		bookmarksopen=true,
		bookmarksnumbered=true,
		linkcolor=black,
		urlcolor=SkyBlue2,
		plainpages=false,
		pagebackref,
		citecolor=black,
		hypertexnames=true,
		pdfauthor={Carolin Julia Wortmann},
		pdfborderstyle={/S/U},
		linkbordercolor=SkyBlue2,
		colorlinks=false,
		backref=false]{hyperref}
	\hypersetup{final}
	
	
% :::::::::::::::::::::::::::::::::::::::::::::::::::::::::::
% :::::::::::::::::::: Bibliography :::::::::::::::::::::::::
% :::::::::::::::::::::::::::::::::::::::::::::::::::::::::::

	\usepackage[%
		backend=biber,
		style=numeric,
		natbib,
		maxbibnames=2,
		sortcites,
		hyperref,
		url=false, 
		doi=true,
		eprint=false
	]{biblatex}
	\addbibresource{literatur/literatur.bib}
	\DefineBibliographyStrings{ngerman}{% 
		andothers = {et\addabbrvspace al\adddot},             
	}	


% :::::::::::::::::::::::::::::::::::::::::::::::::::::::::::
% :::::::::::::::::: Graphics/Floats ::::::::::::::::::::::::
% :::::::::::::::::::::::::::::::::::::::::::::::::::::::::::

	\usepackage[final]{graphicx}							% More graphics options
	\graphicspath{ {./img/} }								% Filepath for \includegraphics{imagefile}
	\usepackage[labelformat=simple]{subcaption}
	\usepackage{caption}
	\captionsetup{font=small}
	\renewcommand\thesubfigure{(\alph{subfigure})}
	\usepackage{float}
	\usepackage{rotating}


% :::::::::::::::::::::::::::::::::::::::::::::::::::::::::::
% ::::::::::::::::::::: Fonts etc. ::::::::::::::::::::::::::
% :::::::::::::::::::::::::::::::::::::::::::::::::::::::::::
	\usepackage[T1]{fontenc}
	\usepackage{dsfont}
	\usepackage{textcomp} 
	\usepackage[babel=true,final,tracking=smallcaps]{microtype} % MAJOR improvement to text using pdflatex
	\DisableLigatures{encoding = T1, family = tt* } 		% Disable ligatures for monospace fonts
	\usepackage{ellipsis}
	
	\renewcommand{\sfdefault}{pag}							% Set default serif font for \sffamily or \textsf
	\renewcommand{\rmdefault}{cmr}							% Set default roman font for \rmfamily or \textrm


% :::::::::::::::::::::::::::::::::::::::::::::::::::::::::::
% ::::::::::::::::::::: ToDo Notes ::::::::::::::::::::::::::
% :::::::::::::::::::::::::::::::::::::::::::::::::::::::::::

	\usepackage[disable]{todonotes} 						% Add \todo{text} Notes to document


% :::::::::::::::::::::::::::::::::::::::::::::::::::::::::::
% ::::::::::::::::::::::: Maths :::::::::::::::::::::::::::::
% :::::::::::::::::::::::::::::::::::::::::::::::::::::::::::

	\usepackage{mathtools}									% Extends amsmath
	\usepackage{amssymb}									% Symbols
	\usepackage{wasysym}									% More symbols
	\usepackage{amsthm}										% Allows use of theorems etc.
	\usepackage[bigdelims]{newtxmath}						% Modern font in math mode
	\allowdisplaybreaks										% Allow pagebreaks in math mode
	\usepackage{bm}											% Allow bold math font


% :::::::::::::::::::::::::::::::::::::::::::::::::::::::::::
% :::::::::::::::::::::::: Tikz :::::::::::::::::::::::::::::
% :::::::::::::::::::::::::::::::::::::::::::::::::::::::::::

	\usepackage{tikz}										% Use tikz to draw graphics
	\usepackage{tikz-cd}									% Commutative diagrams
	\usetikzlibrary{arrows.meta}			
	\usetikzlibrary{calc}					
	\tikzset{>=Latex}										% Set arrow head


% :::::::::::::::::::::::::::::::::::::::::::::::::::::::::::
% ::::::::::::::::::: Pagestyle etc :::::::::::::::::::::::::
% :::::::::::::::::::::::::::::::::::::::::::::::::::::::::::

	\usepackage{scrlayer-scrpage}							% Gives interface similar to old scrpage2
	\pagestyle{scrheadings}
	\usepackage[%											% Set borders
		top=1.5cm, 
		bottom=2cm, 
		left=3.5cm, 
		right=3.5cm, 
		includeheadfoot]{geometry}
	
	\clearscrheadfoot 
	\setkomafont{pagehead}{\bfseries }						% Font option header
	\setkomafont{chapter}{\Large}							% Font option chapter title
	\setkomafont{section}{\large}							% Font option section title
	\setkomafont{subsection}{\large}						% Font option subsection title
	\setkomafont{paragraph}{\rmfamily\normalsize}			% Font option paragraph title		
	\setkomafont{pagefoot}{\normalfont\footnotesize}		% Font option footer
	
	\renewcommand{\chapterpagestyle}{scrheadings}			% First page of chapter should have same format
	\lehead{\scshape\leftmark} 								% Show chapter title on top outer of left page
	\rohead{\rightmark}										% Show section title on top outer of right page
	\ofoot[\pagemark]{\pagemark}							% Page number on outer side
	\raggedbottom											% Set ragged bottom
	
	
% :::::::::::::::::::::::::::::::::::::::::::::::::::::::::::
% :::::::::::::::::::::: Spacings :::::::::::::::::::::::::::
% :::::::::::::::::::::::::::::::::::::::::::::::::::::::::::
	
	\RedeclareSectionCommand[% 								% Set chapter spacing (before/after title)
	afterindent=false,
	beforeskip=0pt,
	afterskip=.5\baselineskip]{chapter}
	\RedeclareSectionCommand[%								% Set section spacing (before/after title)
	afterindent=false,
	beforeskip=\baselineskip,
	afterskip=.3\baselineskip]{section}
	\RedeclareSectionCommand[%								% Set subsection spacing (before/after title)
	afterindent=false,
	beforeskip=.75\baselineskip,
	afterskip=.3\baselineskip]{subsection}
	
	\usepackage{setspace}									% Advanced spacing options
	\setstretch{1.15}										% Set spacing between lines
	\setlength{\parskip}{0\baselineskip}					% Set skip of new paragraphs to zero
	
	
% :::::::::::::::::::::::::::::::::::::::::::::::::::::::::::
% :::::::::::::::::: Tables and Lists :::::::::::::::::::::::
% :::::::::::::::::::::::::::::::::::::::::::::::::::::::::::

	\usepackage{multicol}									% Merge columns
	\usepackage{multirow}									% Merge rows
	\usepackage[shortlabels]{enumitem}						% Control layout of itemize, description and enumerate
	\setlist{itemsep=0pt}									% Set item separation for all those environments
	\setlist[enumerate]{font=\sffamily}						% SF font for enum items		
	\setlist[itemize]{label=$\triangleright$}				% Triangular label for itemize
	\usepackage{tabularx}									% Advanced options for tables


% :::::::::::::::::::::::::::::::::::::::::::::::::::::::::::
% ::::::::::::::::::::: Pseudocode ::::::::::::::::::::::::::
% :::::::::::::::::::::::::::::::::::::::::::::::::::::::::::

	\usepackage{algorithmicx}								% Use algorithmics to style algorithms
	\usepackage{algorithm}									% Basic display of algorithms
	\floatname{algorithm}{Algorithmus} 						% German identifier
	\makeatletter											% Algorithms should be numbered as chapternumber.number
		\renewcommand{\thealgorithm}{\arabic{chapter}.\arabic{algorithm}}
		\@addtoreset{algorithm}{chapter}		
	\makeatother
	\usepackage[noend]{algpseudocode}						% No endif/endfor/etc statements
	\makeatletter											% German title for \listofalgorithms
		\renewcommand{\listalgorithmname}{Algorithmenverzeichnis}
	\makeatother
	\usepackage{eqparbox}
%	\renewcommand{\algorithmiccomment}[1]{\hfill // #1}			
%	\renewcommand{\Comment}[2][.6\linewidth]{%				% Display comments as "// [...]"
%		\leavevmode\hfill\makebox[#1][l]{//~#2}}
	\renewcommand\algorithmiccomment[1]{%
		\hfill {\color{darkgrey} // \ \eqparbox{Comment}{#1}}%
	}


% :::::::::::::::::::::::::::::::::::::::::::::::::::::::::::
% :::::::::::::::: Source Code Highlighting :::::::::::::::::
% :::::::::::::::::::::::::::::::::::::::::::::::::::::::::::
% I highly recommend to have a look at the minted documentation to select the formatting and style
% options you like best and check out the paramters to use.

	\usepackage[newfloat]{minted}							% Use nofloat package for captions
	\setminted{%											% Sets formatting for code blocks 
		style=default,										% Another nice option: vs		
		fontsize=\small,
		breaklines,
		breakanywhere=false,
		breakbytoken=false,
		breakbytokenanywhere=false,
		breakafter={.,},
		autogobble,
		numbersep=3mm,
		tabsize=4,
		linenos,
		frame=lines
	}
	\setmintedinline{%										% Sets inline format
		style=default,
		fontsize=\normalsize,
		numbers=none,
		numbersep=12pt,
		tabsize=4,
	}

	% Surround minted block with \begin{longlisting} ... \end{longlisting} to allow pagebreaks
	\newenvironment{longlisting}{\captionsetup{type=listing}}{}
	
	% Java
	% Use \inputJava[parameters]{filepath}{caption}{label} or \inputJava{filepath}{caption}{label} for blocks using files
	% Use \begin{JavaCode}{caption}{label} \end{JavaCode} (you may use optional parameters for minted as \begin{JavaCode}[optional parameters]{caption}{label} \end{JavaCode} )
	% Use \inlineJava{code} for inline snippets
	\DeclareDocumentCommand{\inputJava}{o m m m}{%	
		\IfNoValueTF{#1}{\begin{longlisting}\inputminted{java}{#2}\captionof{listing}{#3}\label{list:#4}\end{longlisting}}{% 
			\begin{longlisting}
			\inputminted[#1]{java}{#2}%
			\captionof{listing}{#3}
			\label{list:#4}
			\end{longlisting}
			
		}%
	}
	\newmintinline[inlineJava]{java}{}
	\newenvironment{JavaCode}[3][]
	{\VerbatimEnvironment\begin{longlisting}\def\codecaption{#2}\def\codelabel{#3}\begin{minted}[#1]{c}}{\end{minted}\captionof{listing}{\codecaption}\label{list:\codelabel}\end{longlisting}}
	

	% C
	% Usage similar to Java
	\DeclareDocumentCommand{\inputC}{o m m m}{%	
		\IfNoValueTF{#1}{\begin{longlisting}\inputminted{c}{#2}\captionof{listing}{#3}\label{list:#4}\end{longlisting}}{% 
			\begin{longlisting}
				\inputminted[#1]{c}{#2}%
				\captionof{listing}{#3}
				\label{list:#4}
			\end{longlisting}
		}%
	}
	\newmintinline[inlineC]{c}{}
	\newenvironment{CCode}[3][]
	{\VerbatimEnvironment\begin{longlisting}\def\codecaption{#2}\def\codelabel{#3}\begin{minted}[#1]{c}}{\end{minted}\captionof{listing}{\codecaption}\label{list:\codelabel}\end{longlisting}}
	
	% C++
	% Usage similar to Java
	\DeclareDocumentCommand{\inputCpp}{o m m m}{%	
		\IfNoValueTF{#1}{\begin{longlisting}\inputminted{cpp}{#2}\captionof{listing}{#3}\label{list:#4}\end{longlisting}}{% 
			\begin{longlisting}
				\inputminted[#1]{cpp}{#2}%
				\captionof{listing}{#3}
				\label{list:#4}
			\end{longlisting}
		}%
	}
	\newmintinline[inlineCpp]{cpp}{}
	\newenvironment{CppCode}[3][]
	{\VerbatimEnvironment\begin{longlisting}\def\codecaption{#2}\def\codelabel{#3}\begin{minted}[#1]{cpp}}{\end{minted}\captionof{listing}{\codecaption}\label{list:\codelabel}\end{longlisting}}
	
	% Any other language: minted supports most languages
	% For code blocks: \inputminted[parameter]{Sprache}{Dateipfad}, first two are optional
	
	\newcommand{\setLineNumbering}[1]{\renewcommand\theFancyVerbLine{\ifnum\value{FancyVerbLine}=1
			\setcounter{FancyVerbLine}{#1}%
			\rmfamily\tiny\arabic{FancyVerbLine}%
			\else
			\rmfamily\tiny\arabic{FancyVerbLine}%
			\fi
	}}
	
	
% :::::::::::::::::::::::::::::::::::::::::::::::::::::::::::::::::::::::::::::::::::::::::::::::::::::::::::::
% :::::::::::::::::::::::::::::::::::::::::::::::::::::::::::::::::::::::::::::::::::::::::::::::::::::::::::::

% :::::::::::::::::::::::::::::::::::::::::::::::::::::::::::
% :::::::::::::::: Theorems/Definitions/etc :::::::::::::::::
% :::::::::::::::::::::::::::::::::::::::::::::::::::::::::::

	\theoremstyle{plain}										% Style for anything you're going to prove	
	\newtheorem{theorem}[algorithm]{Theorem} 					% Use as environment \begin{theorem} ... \end{theorem}
	\newtheorem{lemma}[algorithm]{Lemma}
	\newtheorem{satz}[algorithm]{Satz} 
	\newtheorem{korollar}[algorithm]{Korollar} 
	
	\theoremstyle{definition}									% Style for definitions etc
	\newtheorem{definition}[algorithm]{Definition} 
	\newtheorem{bemerkung}[algorithm]{Bemerkung} 
	\newtheorem{notation}[algorithm]{Notation} 
	%\newtheorem{beweis}[algorithm]{Beweis} 
	
	% Numbering is dependant on algorithm numbering, so chapternumber.number. This means that algorithms, code,
	% definitions, theorems, etc. will use the same counter, which is reset for each chapter.

% :::::::::::::::::::::::::::::::::::::::::::::::::::::::::::
% :::::::::::::::::::::: Names etc.  ::::::::::::::::::::::::
% :::::::::::::::::::::::::::::::::::::::::::::::::::::::::::

	\newcommand{\tAuthor}{Name}
	\newcommand{\tMail}{Mail}
	\newcommand{\tStudentId}{123456}
	\newcommand{\tTitle}{Titel}
	\newcommand{\tReviewer}{Name}
	\newcommand{\tReviewerMail}{Mail}
	\newcommand{\tReviewerInstitute}{Institute}
	\newcommand{\tSecondAdviser}{Name}
	\newcommand{\tSecondAdviserMail}{Mail}
	\newcommand{\tSecondReviewer}{Name}
	\newcommand{\tSecondReviewerInstitute}{Institute}


	\def\blankpage{%
		\clearpage%
		\thispagestyle{empty}%
		\addtocounter{page}{-1}%
		\null%
		\clearpage}
	\makeatletter
	\let\newtitle\@title								
	\let\newauthor\@author
	\makeatother	
\usepackage{lipsum}


\renewcommand{\tAuthor}{Vorname Nachname}					% Eigener Name
\renewcommand{\tMail}{mail@wwu.de}							% Eigene Mailadresse
\renewcommand{\tTitle}{Titel der Arbeit}					% Titel der Arbeit
\renewcommand{\tReviewer}{Prof. Dr. Max Mustermann}			% Erstbetreuer/-gutachter
\renewcommand{\tReviewerMail}{max.mustermann@wwu.de}		% Erstbetreuer/-gutachter Mail
\renewcommand{\tReviewerInstitute}{Informatik}				% Erstbetreuer Institut
\renewcommand{\tSecondAdviser}{Jane Doe}					% Zweitbetreuer
\renewcommand{\tSecondAdviserMail}{jane.doe@wwu.de}			% Zweitbetreuer Mail
\renewcommand{\tSecondReviewer}{Prof. Dr. John Doe}			% Zweitgutachter
\renewcommand{\tSecondReviewerInstitute}{Informatik}		% Zweitgutachter Institut
\newdate{submissiondate}{13}{01}{2020}						% Abgabedatum



\begin{document}


% :::::::::::::::::::::::::::::::::::::::::::::::::::::::::::
% ::::::::::: Titelseite, Inhaltsverzeichnis, etc :::::::::::
% :::::::::::::::::::::::::::::::::::::::::::::::::::::::::::

	% Titelseite 
	\pagestyle{empty}
		
\newgeometry{left=2.5cm,right=2.5cm} 
\begin{centering}
%\vspace*{30pt} 
\includegraphics[width=5cm]{wwulogo}

%{\Large \textsc{Westfälische Wilhelms-Universität Münster}}

%\includegraphics[width=.4\linewidth]{Logos/WWULogo}

\vspace{2cm} 
{
\begin{spacing}{0.15}
{\rule{\linewidth}{.3mm}} \newline  
{\rule{\linewidth}{.2mm}} \\[0.7cm]
\end{spacing}
}

{\LARGE\sffamily
	\textbf{\tTitle}\\[0.7cm]
}
{
	\begin{spacing}{0.15}
		{\rule{\linewidth}{.2mm}} \newline  
		{\rule{\linewidth}{.3mm}} 
	\end{spacing}
}

\vspace{45pt}

{\normalsize\sffamily
	Zur Erlangung des akademischen Grades \\ \textit{Bachelor of Science}\\ im Studiengang Informatik\\[3cm]
}

\hspace*{30pt}
\begin{minipage}[t]{.9\linewidth}
\begin{minipage}[t]{.34\textwidth}
{\sffamily	\textit{Vorgelegt von:} \\
{\large  \sffamily	 \textbf{\tAuthor}}}\\
{\normalsize\sffamily Matrikelnummer: \tStudentId\\
\tMail}\\[1cm]
\end{minipage} 
\hspace{110pt} 
			\begin{minipage}[t]{.38\textwidth}
					\textit{\sffamily Betreuer:}\\	
			\textbf{\large \sffamily \tReviewer}	\\
				{\normalsize\sffamily \tReviewerMail}\\[0.1cm] 
			\textbf{\large\sffamily  		\tSecondAdviser}\\
				{\normalsize\sffamily   \tSecondAdviserMail}\\			
			\end{minipage}
\end{minipage}
\ \\[2cm]
                               
{\sffamily
	\begin{tabular}{lp{0.5cm}l}
		Erstgutachter:& &\tReviewer, Institut für \tReviewerInstitute \\	Zweitgutachter:& &\tSecondReviewer, Institut für \tSecondReviewerInstitute
	\end{tabular}
	\\[1.5cm]
}

{\sffamily
Münster, \displaydate{submissiondate}
}
\vfill
\end{centering}
\restoregeometry

		\blankpage 
	
	%\listoftodos 											% Anzeigen aller \todo{text} am Anfang als Liste
	
	
	\pagenumbering{Roman} 									% Römische Seitennummern für den Anfangsteil
	\pagestyle{empty} 										% Abstract mit leerer Seite
	%\pagestyle{empty}


\vspace*{150pt}


\section*{Vorwort}
\noindent 
\lipsum[1]
\vfill
	
	\pagestyle{scrheadings} 								% Seitenstil innerhalb der Arbeit und für das Inhaltsverzeichnis
	
	% Inhaltsverzeichnis
	%\setcounter{tocdepth}{1} 								% Falls man die Level des TOC anpassen möchte 
	\tableofcontents
	%\addtocontents{toc}{~\hfill\textbf{Seite}\par} 		% Anzeige von "Seite" über Seitenzahlen
	
	\blankpage  
	
	\pagestyle{scrheadings}
	
	\pagenumbering{arabic}


% :::::::::::::::::::::::::::::::::::::::::::::::::::::::::::
% :::::::::::::::::: Eigentliche Arbeit :::::::::::::::::::::
% :::::::::::::::::::::::::::::::::::::::::::::::::::::::::::

	% Die Hauptkapitel der Arbeit
	\chapter{Beispiele}
Nachfolgend zeige ich einige kurze Ausschnitte meiner Bachelorarbeit, um beispielhaft zu zeigen, wie bestimmte Umgebungen genutzt werden können. Man kann in allen Fälen noch viel mehr machen, aber so entsteht vielleicht eine Idee. 

\section{Algorithmen}
Ein wichtiger Aspekt ist oft Pseudocode. Hierzu gibt es eine Vielzahl von packages, ich selbst nutze \texttt{algorithmicx} in der \texttt{algpseudocode}-Variante. Die Bedienung ist recht intutitiv. Nachfolgend ein Beispiel zur diskreten Fréchet-Distanz auf Trajektorien (zeitlich geordneten Punktfolgen).
\begin{algorithm}
	\caption{zur Berechnung der diskreten Fréchet-Distanz zwischen zwei Trajektorien $A=(a_1,\dots,a_n)$ und $B=(b_1,\dots,b_m)$ mit $1\leq n,m$.}
	\label{alg:dfrechet}
	\begin{algorithmic}[1] % l wenn keine Nummern
		\Require Es gilt $|A|,|B| \geq 1$.
		\Function {discreteFréchet}{Trajektorie $A$,Trajektorie $B$}
		\State Erzeuge Feld $D[1,\dots, n][1,\dots, m]$
		\For{$i:=1$ to $n$}
		\For {$j:=1$ to $m$} 
		\If{$i=j=1$}
		\State $D[1][1]:=d(a_1,b_1)$ 	\Comment{Basisfall}
		\ElsIf{$i=1$} \Comment{zweite Bedingung}
		\State $D[i][j]:=\max\{d(a_i,b_j),D[i][j-1]\}$
		\ElsIf{$j=1$} \Comment{dritte Bedingung}
		\State $D[i][j]:=\max\{d(a_i,b_j),D[i-1][j]\}$
		\Else \Comment{$i,j>1$}
		\State $D[i][j]:=\max\{d(a_i,b_j),\min\{D[i][j-1],D[i-1][j],D[i-1][j-1]\}\}$
		\EndIf 
		\EndFor 
		\EndFor
		\State 	\Return $D[n][m]$ 
		\EndFunction 
	\end{algorithmic}
\end{algorithm}

\section{Code-Beispiele}
Für Code-Beispiele nutze ich \texttt{minted}, in der config-Datei habe ich Makros für Java, C und C++ hinzugefügt. Das geht entweder über ausgelagerte Dateien oder über direktes Einfügen des Codes.  Labels werden automatisch mit dem Prefix "list:" versehen. 
Dieser Code liefert die nachfolgende Ausgabe:
	\begin{minted}{tex}
		\begin{JavaCode}{Dies ist eine Beschreibung.}{label}
			// Simple implementation of recursive equation for coupling distance.
			for (int i = 0; i < pointsA.length; i++) {
				for (int j = 0; j < pointsB.length; j++) {
					double deltaCurrent =
					delta.getDistance(pointsA[i].getxValue(), pointsA[i].getyValue(), pointsB[j].getxValue(), pointsB[j].getyValue());
					if (j == 0 && i == 0) {
						resultMatrix[i][j] = deltaCurrent;
					} else if (i == 0) {
						resultMatrix[i][j] = distanceOperator(deltaCurrent, resultMatrix[i][j-1]);
					} else if (j == 0) {
						resultMatrix[i][j] = distanceOperator(deltaCurrent, resultMatrix[i-1][j]);
					} else {
						resultMatrix[i][j] =
						distanceOperator(deltaCurrent, getMin(resultMatrix[i-1][j], resultMatrix[i][j-1], resultMatrix[i-1][j-1]));
					}
				}
			}
		\end{JavaCode}
	\end{minted}
{\setLineNumbering{5}
	\begin{JavaCode}{Dies ist eine Beschreibung.}{label}
		// Simple implementation of recursive equation for coupling distance.
		for (int i = 0; i < pointsA.length; i++) {
			for (int j = 0; j < pointsB.length; j++) {
				double deltaCurrent =
				delta.getDistance(pointsA[i].getxValue(), pointsA[i].getyValue(), pointsB[j].getxValue(), pointsB[j].getyValue());
				if (j == 0 && i == 0) {
					resultMatrix[i][j] = deltaCurrent;
				} else if (i == 0) {
					resultMatrix[i][j] = distanceOperator(deltaCurrent, resultMatrix[i][j-1]);
				} else if (j == 0) {
					resultMatrix[i][j] = distanceOperator(deltaCurrent, resultMatrix[i-1][j]);
				} else {
					resultMatrix[i][j] =
					distanceOperator(deltaCurrent, getMin(resultMatrix[i-1][j], resultMatrix[i][j-1], resultMatrix[i-1][j-1]));
				}
			}
		}
	\end{JavaCode}
	Man kann dabei die Zeilennummern mit dem optionalen Parameter \enquote{linenos=false} ausstellen oder mit \enquote{firstline=151} bzw. \enquote{lastline=160} manuell einschränken, welchen Bereich des Code-Ausschnittes/der Datei man zeigen möchte (bietet sich vor allem beim Einbinden von Dateien an). Mit dem Makro \texttt{\setLineNumbering{<value>}} könnt ihr den Startwert der Zeilennummern setzen. \textbf{Wenn ihr das außerhalb einer Gruppe/Umgebung setzt, ist die Änderung global. Also am besten den Befehl und den COde, auf den es sich auswirken soll, mit geschweiften Klammern versehen :)}}
	
	Das Einbinden von Code aus Dateien funktioniert analog mit 
	\begin{minted}[linenos=false]{tex}
	\inputJava[parameters]{filepath}{caption}{label}
	\end{minted}
	Dabei sind die Parameter wieder optional. Die gleichen Makros sind auch für C und C++ angelegt (ersetzt \enquote{Java} durch \enquote{C} bzw. \enquote{C++}). 

\section{Grafiken}

\section{Theoreme, Sätze, Lemmata, Definitionen}

\section{Grafiken}

\section{Einige nützliche Tools}
	
	Dies ist ein Beispiel, wie man einfach zitieren kann 
	\citep{smith2009}, natürlich funktionieren auch bestimmte Kapitel oder Abschnitte \citep[Kapitel 1]{smith2010}.
	
	\chapter{Zitationen}
Die ist ein Beispiel, wie man zitiert\citep{smith2009}. Natürlich geht das auch mit Angabe des Kapitels\citep[Kapitel 1]{smith2010}.



% :::::::::::::::::::::::::::::::::::::::::::::::::::::::::::
% ::::::::::::::: Verzeichnisse, Anhang :::::::::::::::::::::
% :::::::::::::::::::::::::::::::::::::::::::::::::::::::::::

	% Literaturverzeichnis
	\printbibliography[heading=bibintoc]
	
	
	
	\pagestyle{plain}
	%% Abbildungsverzeichnis
	%\addcontentsline{toc}{chapter}{\listfigurename}
	%\listoffigures
	%
	%%% Tabellenverzeichnis
	%%\addcontentsline{toc}{chapter}{\listtablename}
	%%\listoftables
	%% Algorithmenverzeichnis
	%\addcontentsline{toc}{chapter}{\listalgorithmname}
	
	%\listofalgorithms
	
	% Ggf Anhang


% :::::::::::::::::::::::::::::::::::::::::::::::::::::::::::
% !!!!!!!!!! Eidesstattliche Erklärung, WICHTIG !!!!!!!!!!!!!
% :::::::::::::::::::::::::::::::::::::::::::::::::::::::::::
	
	\chapter*{Eidesstattliche Erklärung}
\pagestyle{plain}
\rohead{ } % Header anpassen (=keiner)
\lehead{ } 
% Die Aktuelle Version der Eidesstättlichen Erklärung kann beim zuständigen Prüfungsamts gefunden werden.
Hiermit versichere ich, dass die vorliegende Arbeit über \textit{\enquote{\tTitle}} selbstständig verfasst worden ist, dass keine anderen Quellen und Hilfsmittel als die angegebenen benutzt worden sind und dass die Stellen der Arbeit, die anderen Werken – auch elektronischen Medien – dem Wortlaut oder Sinn nach entnommen wurden, auf jeden Fall unter Angabe der Quelle als Entlehnung kenntlich gemacht worden sind.

\vspace{1cm}

\parbox{20em}{\hrulefill}

\tAuthor, Münster, \displaydate{submissiondate}

\vspace{1cm}

Ich erkläre mich mit einem Abgleich der Arbeit mit anderen Texten zwecks Auffindung von Übereinstimmungen sowie mit einer zu diesem Zweck vorzunehmenden Speicherung der Arbeit in eine Datenbank einverstanden.

\vspace{1cm}

\parbox{20em}{\hrulefill}

\tAuthor, Münster,  \displaydate{submissiondate}	




\end{document}
