% Author: Carolin Wortmann
\documentclass[%
	paper=a4,
	fontsize=11pt,
	ngerman,
	twoside,
	toc=listof, 
	]{scrbook}

% :::::::::::::::::::::::::::::::::::::::::::::::::::::::::::
% ::::::::::::::::: Encoding/Language :::::::::::::::::::::::
% :::::::::::::::::::::::::::::::::::::::::::::::::::::::::::

	\usepackage{scrtime}	
	\usepackage{datetime}									% To set date	
	\usepackage{etex}
	\usepackage{shellesc}									% Compile using -shell-escape!
	\usepackage[utf8]{inputenc}								% UTF8-encoding
	\usepackage{babel}										% German typographical rules 
	\usepackage[german=quotes]{csquotes}					% German quotes 
	
	
% :::::::::::::::::::::::::::::::::::::::::::::::::::::::::::
% :::::::::::::::::::::: Colours ::::::::::::::::::::::::::::
% :::::::::::::::::::::::::::::::::::::::::::::::::::::::::::
	
	\usepackage[usenames,x11names,final]{xcolor}
	\definecolor{fbblau}{HTML}{3078AB}
	\definecolor{mediumgrey}{gray}{.65}
	\definecolor{darkgrey}{gray}{.35}	
	\definecolor{stdblue}{HTML}{00387A}
	\definecolor{stdred}{HTML}{900000}

	
% :::::::::::::::::::::::::::::::::::::::::::::::::::::::::::
% ::::::::::::::::::: Hyperrefs/URLS ::::::::::::::::::::::::
% :::::::::::::::::::::::::::::::::::::::::::::::::::::::::::	
	
	\usepackage{url}
	\usepackage[%
		hidelinks,
		pdfpagelabels,
		bookmarksopen=true,
		bookmarksnumbered=true,
		linkcolor=black,
		urlcolor=SkyBlue2,
		plainpages=false,
		pagebackref,
		citecolor=black,
		hypertexnames=true,
		pdfauthor={Carolin Julia Wortmann},
		pdfborderstyle={/S/U},
		linkbordercolor=SkyBlue2,
		colorlinks=false,
		backref=false]{hyperref}
	\hypersetup{final}
	
	
% :::::::::::::::::::::::::::::::::::::::::::::::::::::::::::
% :::::::::::::::::::: Bibliography :::::::::::::::::::::::::
% :::::::::::::::::::::::::::::::::::::::::::::::::::::::::::

	\usepackage[%
		backend=biber,
		style=numeric,
		natbib,
		maxbibnames=2,
		sortcites,
		hyperref,
		url=false, 
		doi=true,
		eprint=false
	]{biblatex}
	\addbibresource{literatur/literatur.bib}
	\DefineBibliographyStrings{ngerman}{% 
		andothers = {et\addabbrvspace al\adddot},             
	}	


% :::::::::::::::::::::::::::::::::::::::::::::::::::::::::::
% :::::::::::::::::: Graphics/Floats ::::::::::::::::::::::::
% :::::::::::::::::::::::::::::::::::::::::::::::::::::::::::

	\usepackage[final]{graphicx}							% More graphics options
	\graphicspath{ {./img/} }								% Filepath for \includegraphics{imagefile}
	\usepackage[labelformat=simple]{subcaption}
	\usepackage{caption}
	\captionsetup{font=small}
	\renewcommand\thesubfigure{(\alph{subfigure})}
	\usepackage{float}
	\usepackage{rotating}


% :::::::::::::::::::::::::::::::::::::::::::::::::::::::::::
% ::::::::::::::::::::: Fonts etc. ::::::::::::::::::::::::::
% :::::::::::::::::::::::::::::::::::::::::::::::::::::::::::
	\usepackage[T1]{fontenc}
	\usepackage{dsfont}
	\usepackage{textcomp} 
	\usepackage[babel=true,final,tracking=smallcaps]{microtype} % MAJOR improvement to text using pdflatex
	\DisableLigatures{encoding = T1, family = tt* } 		% Disable ligatures for monospace fonts
	\usepackage{ellipsis}
	
	\renewcommand{\sfdefault}{pag}							% Set default serif font for \sffamily or \textsf
	\renewcommand{\rmdefault}{cmr}							% Set default roman font for \rmfamily or \textrm


% :::::::::::::::::::::::::::::::::::::::::::::::::::::::::::
% ::::::::::::::::::::: ToDo Notes ::::::::::::::::::::::::::
% :::::::::::::::::::::::::::::::::::::::::::::::::::::::::::

	\usepackage[disable]{todonotes} 						% Add \todo{text} Notes to document


% :::::::::::::::::::::::::::::::::::::::::::::::::::::::::::
% ::::::::::::::::::::::: Maths :::::::::::::::::::::::::::::
% :::::::::::::::::::::::::::::::::::::::::::::::::::::::::::

	\usepackage{mathtools}									% Extends amsmath
	\usepackage{amssymb}									% Symbols
	\usepackage{wasysym}									% More symbols
	\usepackage{amsthm}										% Allows use of theorems etc.
	\usepackage[bigdelims]{newtxmath}						% Modern font in math mode
	\allowdisplaybreaks										% Allow pagebreaks in math mode
	\usepackage{bm}											% Allow bold math font


% :::::::::::::::::::::::::::::::::::::::::::::::::::::::::::
% :::::::::::::::::::::::: Tikz :::::::::::::::::::::::::::::
% :::::::::::::::::::::::::::::::::::::::::::::::::::::::::::

	\usepackage{tikz}										% Use tikz to draw graphics
	\usepackage{tikz-cd}									% Commutative diagrams
	\usetikzlibrary{arrows.meta}			
	\usetikzlibrary{calc}					
	\tikzset{>=Latex}										% Set arrow head


% :::::::::::::::::::::::::::::::::::::::::::::::::::::::::::
% ::::::::::::::::::: Pagestyle etc :::::::::::::::::::::::::
% :::::::::::::::::::::::::::::::::::::::::::::::::::::::::::

	\usepackage{scrlayer-scrpage}							% Gives interface similar to old scrpage2
	\pagestyle{scrheadings}
	\usepackage[%											% Set borders
		top=1.5cm, 
		bottom=2cm, 
		left=3.5cm, 
		right=3.5cm, 
		includeheadfoot]{geometry}
	
	\clearscrheadfoot 
	\setkomafont{pagehead}{\bfseries }						% Font option header
	\setkomafont{chapter}{\Large}							% Font option chapter title
	\setkomafont{section}{\large}							% Font option section title
	\setkomafont{subsection}{\large}						% Font option subsection title
	\setkomafont{paragraph}{\rmfamily\normalsize}			% Font option paragraph title		
	\setkomafont{pagefoot}{\normalfont\footnotesize}		% Font option footer
	
	\renewcommand{\chapterpagestyle}{scrheadings}			% First page of chapter should have same format
	\lehead{\scshape\leftmark} 								% Show chapter title on top outer of left page
	\rohead{\rightmark}										% Show section title on top outer of right page
	\ofoot[\pagemark]{\pagemark}							% Page number on outer side
	\raggedbottom											% Set ragged bottom
	
	
% :::::::::::::::::::::::::::::::::::::::::::::::::::::::::::
% :::::::::::::::::::::: Spacings :::::::::::::::::::::::::::
% :::::::::::::::::::::::::::::::::::::::::::::::::::::::::::
	
	\RedeclareSectionCommand[% 								% Set chapter spacing (before/after title)
	afterindent=false,
	beforeskip=0pt,
	afterskip=.5\baselineskip]{chapter}
	\RedeclareSectionCommand[%								% Set section spacing (before/after title)
	afterindent=false,
	beforeskip=\baselineskip,
	afterskip=.3\baselineskip]{section}
	\RedeclareSectionCommand[%								% Set subsection spacing (before/after title)
	afterindent=false,
	beforeskip=.75\baselineskip,
	afterskip=.3\baselineskip]{subsection}
	
	\usepackage{setspace}									% Advanced spacing options
	\setstretch{1.15}										% Set spacing between lines
	\setlength{\parskip}{0\baselineskip}					% Set skip of new paragraphs to zero
	
	
% :::::::::::::::::::::::::::::::::::::::::::::::::::::::::::
% :::::::::::::::::: Tables and Lists :::::::::::::::::::::::
% :::::::::::::::::::::::::::::::::::::::::::::::::::::::::::

	\usepackage{multicol}									% Merge columns
	\usepackage{multirow}									% Merge rows
	\usepackage[shortlabels]{enumitem}						% Control layout of itemize, description and enumerate
	\setlist{itemsep=0pt}									% Set item separation for all those environments
	\setlist[enumerate]{font=\sffamily}						% SF font for enum items		
	\setlist[itemize]{label=$\triangleright$}				% Triangular label for itemize
	\usepackage{tabularx}									% Advanced options for tables


% :::::::::::::::::::::::::::::::::::::::::::::::::::::::::::
% ::::::::::::::::::::: Pseudocode ::::::::::::::::::::::::::
% :::::::::::::::::::::::::::::::::::::::::::::::::::::::::::

	\usepackage{algorithmicx}								% Use algorithmics to style algorithms
	\usepackage{algorithm}									% Basic display of algorithms
	\floatname{algorithm}{Algorithmus} 						% German identifier
	\makeatletter											% Algorithms should be numbered as chapternumber.number
		\renewcommand{\thealgorithm}{\arabic{chapter}.\arabic{algorithm}}
		\@addtoreset{algorithm}{chapter}		
	\makeatother
	\usepackage[noend]{algpseudocode}						% No endif/endfor/etc statements
	\makeatletter											% German title for \listofalgorithms
		\renewcommand{\listalgorithmname}{Algorithmenverzeichnis}
	\makeatother
	\usepackage{eqparbox}
%	\renewcommand{\algorithmiccomment}[1]{\hfill // #1}			
%	\renewcommand{\Comment}[2][.6\linewidth]{%				% Display comments as "// [...]"
%		\leavevmode\hfill\makebox[#1][l]{//~#2}}
	\renewcommand\algorithmiccomment[1]{%
		\hfill {\color{darkgrey} // \ \eqparbox{Comment}{#1}}%
	}


% :::::::::::::::::::::::::::::::::::::::::::::::::::::::::::
% :::::::::::::::: Source Code Highlighting :::::::::::::::::
% :::::::::::::::::::::::::::::::::::::::::::::::::::::::::::
% I highly recommend to have a look at the minted documentation to select the formatting and style
% options you like best and check out the paramters to use.

	\usepackage[newfloat]{minted}							% Use nofloat package for captions
	\setminted{%											% Sets formatting for code blocks 
		style=default,										% Another nice option: vs		
		fontsize=\small,
		breaklines,
		breakanywhere=false,
		breakbytoken=false,
		breakbytokenanywhere=false,
		breakafter={.,},
		autogobble,
		numbersep=3mm,
		tabsize=4,
		linenos,
		frame=lines
	}
	\setmintedinline{%										% Sets inline format
		style=default,
		fontsize=\normalsize,
		numbers=none,
		numbersep=12pt,
		tabsize=4,
	}

	% Surround minted block with \begin{longlisting} ... \end{longlisting} to allow pagebreaks
	\newenvironment{longlisting}{\captionsetup{type=listing}}{}
	
	% Java
	% Use \inputJava[parameters]{filepath}{caption}{label} or \inputJava{filepath}{caption}{label} for blocks using files
	% Use \begin{JavaCode}{caption}{label} \end{JavaCode} (you may use optional parameters for minted as \begin{JavaCode}[optional parameters]{caption}{label} \end{JavaCode} )
	% Use \inlineJava{code} for inline snippets
	\DeclareDocumentCommand{\inputJava}{o m m m}{%	
		\IfNoValueTF{#1}{\begin{longlisting}\inputminted{java}{#2}\captionof{listing}{#3}\label{list:#4}\end{longlisting}}{% 
			\begin{longlisting}
			\inputminted[#1]{java}{#2}%
			\captionof{listing}{#3}
			\label{list:#4}
			\end{longlisting}
			
		}%
	}
	\newmintinline[inlineJava]{java}{}
	\newenvironment{JavaCode}[3][]
	{\VerbatimEnvironment\begin{longlisting}\def\codecaption{#2}\def\codelabel{#3}\begin{minted}[#1]{java}}{\end{minted}\captionof{listing}{\codecaption}\label{list:\codelabel}\end{longlisting}}
	

	% C
	% Use \inputC[parameters]{filepath} or \inputC{filepath} 
	% Use \inlineC{code} for inline snippets
	\DeclareDocumentCommand{\inputC}{o m}{%	
		\IfNoValueTF{#1}{\inputminted{c}{#2}}{% 
			\inputminted[#1]{c}{#2}%
		}%
	}
	\newmintinline[inlineC]{c}{}
	
	% C++
	% Use \inputCpp[parameters]{filepath} or \inputCpp{filepath} 
	% Use \inlineCpp{code} for inline snippets
	\DeclareDocumentCommand{\inputCpp}{o m}{%	
		\IfNoValueTF{#1}{\inputminted{cpp}{#2}}{% 
			\inputminted[#1]{cpp}{#2}%
		}%
	}
	\newmintinline[inlineCpp]{cpp}{}
	
	% Any other language: minted supports most languages
	% For code blocks: \inputminted[parameter]{Sprache}{Dateipfad}, first two are optional
	
	\newcommand{\setLineNumbering}[1]{\renewcommand\theFancyVerbLine{\ifnum\value{FancyVerbLine}=1
			\setcounter{FancyVerbLine}{#1}%
			\rmfamily\tiny\arabic{FancyVerbLine}%
			\else
			\rmfamily\tiny\arabic{FancyVerbLine}%
			\fi
	}}
	
	
% :::::::::::::::::::::::::::::::::::::::::::::::::::::::::::::::::::::::::::::::::::::::::::::::::::::::::::::
% :::::::::::::::::::::::::::::::::::::::::::::::::::::::::::::::::::::::::::::::::::::::::::::::::::::::::::::

% :::::::::::::::::::::::::::::::::::::::::::::::::::::::::::
% :::::::::::::::: Theorems/Definitions/etc :::::::::::::::::
% :::::::::::::::::::::::::::::::::::::::::::::::::::::::::::

	\theoremstyle{plain}										% Style for anything you're going to prove	
	\newtheorem{theorem}[algorithm]{Theorem} 					% Use as environment \begin{theorem} ... \end{theorem}
	\newtheorem{lemma}[algorithm]{Lemma}
	\newtheorem{satz}[algorithm]{Satz} 
	\newtheorem{korollar}[algorithm]{Korollar} 
	
	\theoremstyle{definition}									% Style for definitions etc
	\newtheorem{definition}[algorithm]{Definition} 
	\newtheorem{bemerkung}[algorithm]{Bemerkung} 
	\newtheorem{notation}[algorithm]{Notation} 
	%\newtheorem{beweis}[algorithm]{Beweis} 
	
	% Numbering is dependant on algorithm numbering, so chapternumber.number. This means that algorithms, code,
	% definitions, theorems, etc. will use the same counter, which is reset for each chapter.

% :::::::::::::::::::::::::::::::::::::::::::::::::::::::::::
% :::::::::::::::::::::: Names etc.  ::::::::::::::::::::::::
% :::::::::::::::::::::::::::::::::::::::::::::::::::::::::::

	\newcommand{\tAuthor}{Name}
	\newcommand{\tMail}{Mail}
	\newcommand{\tStudentId}{123456}
	\newcommand{\tTitle}{Titel}
	\newcommand{\tReviewer}{Name}
	\newcommand{\tReviewerMail}{Mail}
	\newcommand{\tReviewerInstitute}{Institute}
	\newcommand{\tSecondAdviser}{Name}
	\newcommand{\tSecondAdviserMail}{Mail}
	\newcommand{\tSecondReviewer}{Name}
	\newcommand{\tSecondReviewerInstitute}{Institute}


	\def\blankpage{%
		\clearpage%
		\thispagestyle{empty}%
		\addtocounter{page}{-1}%
		\null%
		\clearpage}
	\makeatletter
	\let\newtitle\@title								
	\let\newauthor\@author
	\makeatother